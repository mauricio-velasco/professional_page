\documentclass[usepdftitle=false]{beamer}
\usetheme{Berkeley}



\title[]{Taller Vectorial Virtual parte 4: Coordenadas cil\'indricas y esf\'ericas}

\date{Marzo 2020}

\begin{document}

\begin{frame}
  \titlepage
\end{frame}

\begin{frame}{Ejercicio 15}
Para cada una de las siguientes integrales, haga un dibujo de la regi\'on de integraci\'on y evalue la integral cambi\'andola a coordenadas cil\'indricas:
\begin{enumerate}
\item $\iiint_E(x^3+x^2y)dV$ donde $E$ es el s\'olido del primer octante debajo del paraboloide $z=10-x^2-y^2$. 
\item $\iiint_E x^2dV$ donde $E$ es el s\'olido que esta dentro del cilindro $x^2+y^2=1$ encima del plano $z=0$ y debajo del cono $z^2=4x^2+4y^2$.

\end{enumerate}
\end{frame}


\begin{frame}{Ejercicio 16}
Para cada una de las siguientes integrales, haga un dibujo de la regi\'on de integraci\'on y evalue la integral cambi\'andola a coordenadas esf\'ericas:
\begin{enumerate}
\item $\iiint_B(x^2+y^2+z^2)dV$ donde $B$ es la bola de radio $2$ centrada en el origen. 
\item $\iiint_E 1dV$ donde $E$ es la regi\'on dentro de la esfera $x^2+y^2+z^2=4$ que esta debajo del cono $z=\sqrt{x^2+y^2}$.

\end{enumerate}
\end{frame}



\begin{frame}{Ejercicio 17}
Para cada una de las siguientes integrales, haga un dibujo de la regi\'on de integraci\'on y evalue la integral plante\'andola en un sistema de coordenadas conveniente:
\begin{enumerate}
\item $\int_{-3}^3\int_{-\sqrt{9-x^2}}^{\sqrt{9-x^2}}\int_0^{\sqrt{9-x^2-y}}z\sqrt{x^2+y^2+z^2}dzdydx$

\item $\iiint_B\frac{1}{\sqrt{x^2+y^2+z^2}}dV$ donde $B$ es la regi\'on determinada por las condiciones $\frac{1}{2}\leq z\leq 1$ y $x^2+y^2+z^2\leq 1$
\end{enumerate}
\end{frame}



\begin{frame}{Ejercicio 18}
Sea $a$ un n\'umero real positivo. Calcule, mediante integrales:
\begin{enumerate}
\item El volumen de la bola de radio $a$.

\item El volumen del pedazo m\'as peque\~no que se obtiene al cortar una esfera de radio $a$ en cuatro partes mediante dos planos que pasan por el centro y que se intersectan en un di\'ametro formando un \'angulo de $\frac{\pi}{6}$.
\end{enumerate}
Note que las respuestas de ambos problemas deben ser funciones de $a$.
\end{frame}


\begin{frame}{Ejercicio 19}
Sean $a,b,c$ numeros positivos dados y sea $E$ el elipsoide $\frac{x^2}{a^2}+ \frac{y^2}{b^2} + \frac{z^2}{c^2}\leq 1$. Calcule las siguientes integrales:
\begin{enumerate}
\item El volumen de $E$.
\item $\iiint_E \frac{x^2}{a^2}+ \frac{y^2}{b^2} + \frac{z^2}{c^2}dV$
\end{enumerate}
(Sugerencia: Cambie de variables para convertir a $E$ en una bola y luego utilice coordenadas esf\'ericas).
\end{frame}

\begin{frame}{Ejercicio 20} 

Sea $r$ un n\'umero positivo y sea $E$ el s\'olido de intersecci\'on de los dos cilindros $x^2+y^2=r^2$ y $y^2+z^2=r^2$.
\begin{enumerate}
\item Haga un dibujo de $E$
\item Encuentre el volumen de $E$.
\item *OPCIONAL: Puede ud. encontrar el volumen de la intersecci\'on de TRES cilindros, el tercero $x^2+z^2=r^2$.
\end{enumerate}
(Nota: probablemente no sea necesario cambiar de variable para resolver este ejercicio)
\end{frame}


\end{document}