\documentclass[usepdftitle=false]{beamer}
\usetheme{Berkeley}
\begin{document}



\title[]{Taller Vectorial Virtual parte 2: Integrales triples iteradas}

\date{Marzo 2020}

\begin{document}

\begin{frame}
  \titlepage
\end{frame}


\begin{frame}{Ejercicio 6}
\begin{enumerate}
\item Sea $E$ la regi\'on s\'olida encerrada por el paraboloide $y=x^2+z^2$ y el plano $y=4$. Dibuje la regi\'on y escriba la integral sobre $E$ como:

\begin{enumerate}
\item  Una integral iterada con orden $dydzdx$ 
\item  Una integral iterada con orden $dzdxdy$ 
\end{enumerate}

\item Sea $E$ la regi\'on s\'olida encerrada por el paraboloide $x=y^2$ y los planos $x=z$, $z=0$ y $x=1$. Dibuje la regi\'on $E$ y escriba la integral sobre $E$ como:

\begin{enumerate}
\item  Una integral iterada con orden $dzdydx$ 
\item  Una integral iterada con orden $dxdydz$ 
\end{enumerate}


\end{enumerate}

\end{frame}


\begin{frame}{Ejercicio 7}
Para cada una de las siguientes integrales iteradas dibuje la regi\'on de integraci\'on y calcule el valor de la integral
\begin{enumerate}
\item $\int_0^1\int_0^z\int_0^{x+z}xzdydxdz$

\item $\int_0^1\int_{\sqrt{x}}^1\int_0^{1-y}dzdydx$

\end{enumerate}

\end{frame}

\begin{frame}{Ejercicio 8: Masas y densidades}
La funci\'on de densidad de un pozo petrolero potencial (medida en Toneladas por metro c\'ubico) esta dada por
\[\rho(x,y,z)=\begin{cases}
e^{-(x+y+z)}\text{, si $x,y,z\geq 0$}\\
0\text{, de lo contrario}\\
\end{cases}\]
\begin{enumerate}

\item Dibuje los conjuntos de nivel $\frac{1}{e}$, $\frac{1}{e^2}$ y $\frac{1}{e^3}$ de $\rho$ (ac\'a la letra $e$ denota el n\'umero con $log(e)=1$, $e\simeq 2.718$).
\item Dibuje la regi\'on encerrada por los planos $x+z+y\leq 1$ $x=0$, $y=0$, $z=0$ y calcule su masa total asumiendo la funci\'on de densidad de la parte $(1)$.

\end{enumerate}

\end{frame}



\begin{frame}{Ejercicio 9: Probabilidad} La densidad conjunta de tres variables aleatorias $X,Y$ y $Z$ esta dada por 
\[f(x,y,z)=\begin{cases}
Cxyz\text{, si $0\leq x\leq 2$, $0\leq y\leq 1$ y $0\leq z\leq 2$}\\
0\text{, de lo contrario.}
\end{cases}\]
\begin{enumerate}
\item Encuentre el valor de la constante $C$ (El que hace que la integral sobre todo $\mathbb{R}^3$ valga uno)
\item Calcule $\mathbb{P}\{ X\leq 1, Y\leq 1, Z\leq 1\}$ (es decir calcule la integral de $f$ sobre $0\leq x\leq 1$, $0\leq y\leq y$ y $0\leq z\leq 1$)
\item Calcule $\mathbb{P}\{ X+Y+Z\leq 1\}$ (es decir calcule la integral de $f$ sobre la regi\'on donde $x+y+z\leq 1$)

\end{enumerate}

\end{frame}


\end{document}



\begin{frame}{Ejercicio 5}

Evalue las siguientes integrales dobles. No olvide dibujar las regiones de integraci\'on.
\begin{enumerate}
\item $\iint_Dx\cos(y)dA=$ donde $D$ es la regi\'on acotada por las curvas $y=2x^2$ y $y=1$ con $x\geq 0$.
\item $\iint_R (3x-y)dA$ donde $R$ es el c\'irculo de radio $\sqrt{2}$ centrado en el origen.
\end{enumerate}
\end{frame}

\begin{frame}{Ejercicio 2}

Dibuje la regi\'on de integraci\'on de las siguientes integrales iteradas:

\begin{enumerate}
\item $\int_{-1}^0\int_0^{1+y}(1-x+y)dxdy$
\item $\int_{-1}^1\int_{-2|x|}^{|x|} e^{x+y}dydx$
\end{enumerate}

\end{frame}

\begin{frame}{Ejercicio 3}
Eval\'ue las siguientes integrales. ({\it Sugerencia.} Si la integral parece muy dif\'icil, a veces vale la pena cambiar el orden de integraci\'on). No olvide dibujar las regiones de integraci\'on. 

\begin{enumerate}
\item $\int_0^{\sqrt{\pi}}\int_y^{\sqrt{\pi}}\cos(x^2)dxdy=$
\item $\int_0^{1}\int_{\sqrt{x}}^{1}e^{y^3}dydx=$
\end{enumerate}

\end{frame}

\begin{frame}{Ejercicio 4}


Sea $R$ el rect\'angulo $[-1,1]\times [1,3]$. 

\begin{enumerate}
\item Escriba la integral $\iint_R e^{-(x^2+y^2)}dA$ como l\'imite de una doble sumatoria, usando la definición de integral doble. (No olvide escribir f\'ormulas expl\'icitas para el punto $\vec{x_{ij}^*}$ que escogi\'o).

\item OPCIONAL: Usando la parte $(a)$, escriba un programa en python que calcule los primeros $100$, $1000$ y $10000$ t\'erminos de esta suma. Cu\'anto vale aproximadamente la integral? (notar que esta integral no se puede calcular con funciones elementales asi que es necesario aproximarla num\'ericamente).

\end{enumerate}




\end{frame}



\begin{frame}{Ejercicio 5 (Integraci\'on y optimizaci\'on)}

\begin{enumerate}
\item Encuentre los valores m\'aximos y m\'inimos que alcanza la funci\'on $f(x,y)=\frac{1}{x^2+y^2+1}$ en el rect\'angulo $R=[-1,1]\times [1,2]$ (es decir $R$ es el conjunto de $(x,y)$ tales que $-1\leq x\leq 1$ y $ 1\leq y\leq 2$).
\item Utilice el punto anterior para demostrar que
\[1\leq \iint_R \frac{dA}{x^2+y^2+1}\leq 6\]
\end{enumerate}
\end{frame}


\end{document}