\documentclass[usepdftitle=false]{beamer}
\usetheme{Berkeley}



\title[]{Vectorial Virtual -- Taller 3, parte 3: Teoremas de Stokes y Gauss.}

\date{Mayo 2020}

\begin{document}
\maketitle
\begin{frame}{Problema 11: (Teorema de Stokes)}

Sea $H(x,y,z)=x^2y\hat{i} + \frac{x^3}{3}\hat{j} + xy\hat{k}$ y sea $C$ la curva de intersecci\'on del paraboloide hiperb\'olico $z=y^2-x^2$ y el cilindro $x^2+y^2=1$ orientado en la direcci\'on de las manecillas del reloj visto desde arriba:

\begin{enumerate}
\item Grafique el paraboloide hiperb\'olico y el cilindro. Trace la curva de intersecci\'on.
\item Encuentre una parametrizaci\'on para la curva $C$. 
\end{enumerate}

\end{frame}

\begin{frame}{Problema 12: (Teorema de Stokes)}

Calcule el trabajo hecho por el campo de fuerza
\[F(x,y,z) = \left(x^x+z^2, y^y+x^2, z^z+y^2\right)\]
cuando una part\'icula se mueve bajo su influencia alrededor del borde de la parte de la esfera $x^2+y^2+z^2=4$ que se encuentra en el primer octante en direcci\'on contraria a las manecillas del reloj vista desde arriba.

\end{frame}

\begin{frame}{Problema 13: (Teorema de la divergencia)}
Considere el campo vectorial $H$ dador por
\[H(x,y,z)=\left(\sin(x)\cos^2(y), \sin^3(y)\cos^4(z), \sin^5(z)\cos^6(y)\right)\]

\begin{enumerate}
\item Haga un dibujo del campo vectorial
(usando por ejemplo \url{https://www.geogebra.org/m/u3xregNW})

\item Calcule el flujo de $H$ hacia adentro del cubo definido por los seis planos $x=0$, $x=\frac{\pi}{2}$, $y=0$, $y=\frac{\pi}{2}$, $z=0$, $z=\frac{\pi}{2}$.
\end{enumerate}

\end{frame}

\begin{frame}{Problema 14:}

Considere el campo vectorial $F(x,y,z)=(3xy^2, xe^z, z^3)$ y sea $E$ el volumen encerrado por el cilindro $y^2+z^2=1$ 
y los planos $x=-1$ y $x=2$.

\begin{enumerate}
\item Calcule el flujo de $F$ a trav\'es de la frontera de $E$ hacia afuera.

\item La frontera de $E$ consiste de tres partes: la superficie cil\'indrica $S$ y las "tapas" $x=-1$ y $x=2$. Calcule el flujo de $F$ a trav\'es de $S$ hacia afuera.

\end{enumerate}



\end{frame}

\begin{frame}{Problema 15: Teorema de Gauss}
Sea $E$ el campo el\'ectrico generado en $\mathbb{R}^3$ generado por una carga puntual unitaria puesta en el origen 
\[E(\vec{x})=\frac{\vec{x}}{\|\vec{x}\|^3}\]

\begin{enumerate}
\item Haga un dibujo del campo vectorial, 
(por ejemplo usando \url{https://www.geogebra.org/m/u3xregNW}).
\item Verifique que el campo tiene divergencia cero $\nabla\cdot E=0$.
\item Calcule el flujo de $E$ a trav\'es de la esfera unitaria orientada hacia afuera parametrizando la esfera.
\item Calcule el flujo de $E$ hacia adentro del cubo $[-2,2]\times [-2,2]\times [-2,2]$ (sugerencia: Use el Teo de Gauss).
\item Es $E$ un campo conservativo?\end{enumerate}

\end{frame}


\begin{frame}{Problema 16: (Verdadero o Falso)}

Verdadero o Falso: Para cada una de las siguientes afirmaciones de una justificaci\'on (si cree que es $V$) \'o un contraejemplo (si cree que es $F$).

\begin{enumerate}
\item Si $\nabla\cdot H=0$ entonces el campo vectorial $H$ es conservativo.
\item Si $\nabla\times H=0$ entonces el campo vectorial $H$ es conservativo.
\item Si $H=\nabla \times F$ entonces $H$ es conservativo.
\item Si $\nabla\cdot F=0$ entonces el trabajo realizado por $F$ a lo largo de toda curva cerrada es cero.
\item $\nabla \times \nabla u=\vec{0}$ para 
toda funcion escalar diferenciable $u:\mathbb{R}^3\rightarrow \mathbb{R}$.
\item Si $H$ es conservativo entonces ${\rm div}(H)=0$.
\end{enumerate}

\end{frame}
\end{document}


