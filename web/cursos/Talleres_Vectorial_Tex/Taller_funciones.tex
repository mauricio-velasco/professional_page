\documentclass[usepdftitle=false]{beamer}
\usetheme{Berkeley}



\title[]{Vectorial Virtual -- Taller 1, parte 2: Funciones}
\date{}

\begin{document}
\maketitle
\begin{frame}{Problema 1: Conjuntos de nivel}

Dibuje los conjuntos de nivel $-2,-1,0,1,2$
de las siguientes funciones escalares:

\begin{enumerate}

\item $\ell(x)=x^2-1$
\item $r(x,y)=-2x^2-3y^2$
\item $h(x,y)=|x|+|y|$
\item \[g(x,y)=\begin{cases}
1\text{, si $y=x^2$}\\
0\text{, de lo contrario}\\
\end{cases}
\]
\item $t(x,y,z)=2x^2+\tfrac{1}{2}y^2+z^2$
\item $f(x,y,z)= x^2+y^2-z^2$
\end{enumerate}

\end{frame}

\begin{frame}{Problema 2: Gr\'afica de una funci\'on escalar}

Escriba una ecuaci\'on para la la gr\'afica de las siguientes funciones escalares. Haga tambi\'en un dibujo (si esto es posible por razones de dimensi\'on)

\begin{enumerate}
\item $\ell(x)=x^2-1$
\item $r(x,y)=-2x^2-3y^2$
\item $h(x,y)=|x|+|y|$
\item \[g(x,y)=\begin{cases}
1\text{, si $y=x^2$}\\
0\text{, de lo contrario}\\
\end{cases}
\]

\item $t(x,y,z)=2x^2+\tfrac{1}{2}y^2+z^2$
\item $f(x,y,z)= x^2+y^2-z^2$
\end{enumerate}

(Nota: Son las mismas funciones del ejercicio anterior asi que quiz\'as las gr\'aficas de los conjuntos de nivel que ya hizo le ayuden).

\end{frame}


\begin{frame}{Problema 3: L\'imites}

Para cada uno de los siguientes l\'imites demuestre que no existe \'o calcule su valor:

\begin{enumerate}
\item \[\lim_{(x,y)\rightarrow (0,0)} \frac{x^3+y^3}{xy}=\]
\item Sea  $g(x,y)$ la funci\'on dada por
\[g(x,y)=\begin{cases}
1\text{, si $y=x^2$}\\
0\text{, de lo contrario}\\
\end{cases}
\]
$\lim_{(x,y)\rightarrow (0,0)} g(x,y)=$

\item \[\lim_{(x,y)\rightarrow (1,0)} \frac{(x-1)^2+2(x-1)y}{(x-1)y}=\]

\end{enumerate}


\end{frame}

\begin{frame}{Problema 4: Continuidad}

\begin{enumerate}
\item{
\begin{enumerate}
\item {Demuestre que la funci\'on:

\[f(x,y)=\sin(x^2+y^2)-\exp\left(\tan\left(\frac{1}{1+x^2+y^2}\right)\right)\]
es continua en todo $\mathbb{R}^2$ usando un \'arbol de composici\'on.}

\item {Cu\'anto vale $\lim_{(x,y)\rightarrow (0,0)}f(x,y)$? Explique su respuesta.}

\end{enumerate}
}
\item Existe un valor para la constante $c$ que haga que la siguiente funci\'on sea continua en todo el plano? Justifique su respuesta
\[g(x,y)=
\begin{cases}
1+\frac{x^3+y^3}{xy}\text{ si $(x,y)\neq 0$}\\
c\text{, si $(x,y)=(0,0)$}\\
\end{cases}
\]
\end{enumerate}
\end{frame}
\end{document}

\begin{frame}{Problema 1: Masa total}
Calcule la masa de la parte de la superficie parab\'olica $z=x^2+y^2$ con $0\leq z\leq 4$ si la densidad de un punto $(x,y,z)$ de la superficie es igual a $z$.
\end{frame}

\begin{frame}{Problema 2: Centro de masa}
Encuentre las coordenadas del centro de masa de la parte de la esfera (hueca) de radio $10$ ($x^2+y^2+z^2=10$) cuyos puntos satisfacen $x,y,z\geq 0$. Asuma que la densidad de un punto en la esfera es constante.

\end{frame}

\begin{frame} {Problema 3: Momento de Inercia}

Calcule el momento de inercia de una esfera (hueca) de radio $R$ y masa $M$ alrededor de un eje que pasa a trav\'es del centro de la esfera.
\end{frame}

\begin{frame}{Problema 4: }
Una tuber\'ia cil\'indrica con centro el eje $x$ y radio $5$ contiene un fluido viscoso que se mueve siguiento el campo de velocidad $V=\left(e^{-x},0,0\right)$ metros por segundo. Calcule el flujo (en $m^3/sec$) a trav\'es de los siguientes cortes (asumiendo que la orientaci\'on del corte esta dada por un vector con coordenada $x>0$):
\begin{enumerate}
\item Un corte perpendicular a la tuber\'ia.
\item Un corte hecho con el plano $y=x+10$.
\item Un corte contenido en el plano $z=0$.
\end{enumerate} 
\end{frame}

\begin{frame}{Problema 5: }
Sea $\vec{x}=(x,y,z)$ un punto en $\mathbb{R}^3$. Llamemos $F$ al campo vectorial dado por $F(\vec{x})=-\frac{x}{\|x\|^3}$.
Calcule el flujo del campo vectorial $F$ a trav\'es de la esfera de radio 2 centrada en el origen (orientada hacia afuera).

\end{frame}


\end{document}