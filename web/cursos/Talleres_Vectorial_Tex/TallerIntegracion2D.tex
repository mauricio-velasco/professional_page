\documentclass[usepdftitle=false]{beamer}
\usetheme{Berkeley}
\begin{document}



\title[]{Vectorial Virtual: Taller 3, parte 1: Integrales Dobles.}

\date{}

\begin{document}

\begin{frame}
  \titlepage
\end{frame}


\begin{frame}{Ejercicio 1}

Evalue las siguientes integrales dobles. No olvide dibujar las regiones de integraci\'on.
\begin{enumerate}
\item $\iint_Dx\cos(y)dA=$ donde $D$ es la regi\'on acotada por las curvas $y=2x^2$ y $y=1$ con $x\geq 0$.
\item $\iint_R (3x-y)dA$ donde $R$ es el c\'irculo de radio $\sqrt{2}$ centrado en el origen.
\end{enumerate}
\end{frame}

\begin{frame}{Ejercicio 2}

Dibuje la regi\'on de integraci\'on de las siguientes integrales iteradas:

\begin{enumerate}
\item $\int_{-1}^0\int_0^{1+y}(1-x+y)dxdy$
\item $\int_{-1}^1\int_{-2|x|}^{|x|} e^{x+y}dydx$
\end{enumerate}

\end{frame}

\begin{frame}{Ejercicio 3}
Eval\'ue las siguientes integrales. ({\it Sugerencia.} Si la integral parece muy dif\'icil, a veces vale la pena cambiar el orden de integraci\'on). No olvide dibujar las regiones de integraci\'on. 

\begin{enumerate}
\item $\int_0^{\sqrt{\pi}}\int_y^{\sqrt{\pi}}\cos(x^2)dxdy=$
\item $\int_0^{1}\int_{\sqrt{x}}^{1}e^{y^3}dydx=$
\end{enumerate}

\end{frame}

\begin{frame}{Ejercicio 4}


Sea $R$ el rect\'angulo $[-1,1]\times [1,3]$. 

\begin{enumerate}
\item Escriba la integral $\iint_R e^{-(x^2+y^2)}dA$ como l\'imite de una doble sumatoria, usando la definición de integral doble. (No olvide escribir f\'ormulas expl\'icitas para el punto $\vec{x_{ij}^*}$ que escogi\'o).

\item OPCIONAL: Usando la parte $(a)$, escriba un programa en python que calcule los primeros $100$, $1000$ y $10000$ t\'erminos de esta suma. Cu\'anto vale aproximadamente la integral? (notar que esta integral no se puede calcular con funciones elementales asi que es necesario aproximarla num\'ericamente).

\end{enumerate}




\end{frame}



\begin{frame}{Ejercicio 5 (Integraci\'on y optimizaci\'on)}

\begin{enumerate}
\item Encuentre los valores m\'aximos y m\'inimos que alcanza la funci\'on $f(x,y)=\frac{1}{x^2+y^2+1}$ en el rect\'angulo $R=[-1,1]\times [1,2]$ (es decir $R$ es el conjunto de $(x,y)$ tales que $-1\leq x\leq 1$ y $ 1\leq y\leq 2$).
\item Utilice el punto anterior para demostrar que
\[\frac{1}{3}\leq \iint_R \frac{dA}{x^2+y^2+1}\leq 6\]
Puede dar una estimaci\'on m\'as precisa?
\end{enumerate}
\end{frame}


\end{document}