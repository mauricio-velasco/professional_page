\documentclass[usepdftitle=false]{beamer}
\usetheme{Berkeley}



\title[]{Vectorial Virtual -- Taller 3, parte 1: Integrales (de ambos tipos) sobre superficies}

\date{Mayo 2020}

\begin{document}

\begin{frame}{Problema 1: Masa total}
Calcule la masa de la parte de la superficie parab\'olica $z=x^2+y^2$ con $0\leq z\leq 4$ si la densidad de un punto $(x,y,z)$ de la superficie es igual a $z$.
\end{frame}

\begin{frame}{Problema 2: Centro de masa}
Encuentre las coordenadas del centro de masa de la parte de la esfera (hueca) de radio $10$ ($x^2+y^2+z^2=10$) cuyos puntos satisfacen $x,y,z\geq 0$. Asuma que la densidad de un punto en la esfera es constante.

\end{frame}

\begin{frame} {Problema 3: Momento de Inercia}

Calcule el momento de inercia de una esfera (hueca) de radio $R$ y masa $M$ alrededor de un eje que pasa a trav\'es del centro de la esfera.
\end{frame}

\begin{frame}{Problema 4: }
Una tuber\'ia cil\'indrica con centro el eje $x$ y radio $5$ contiene un fluido viscoso que se mueve siguiento el campo de velocidad $V=\left(e^{-x},0,0\right)$ metros por segundo. Calcule el flujo (en $m^3/sec$) a trav\'es de los siguientes cortes (asumiendo que la orientaci\'on del corte esta dada por un vector con coordenada $x>0$):
\begin{enumerate}
\item Un corte perpendicular a la tuber\'ia.
\item Un corte hecho con el plano $y=x+10$.
\item Un corte contenido en el plano $z=0$.
\end{enumerate} 
\end{frame}

\begin{frame}{Problema 5: }
Sea $\vec{x}=(x,y,z)$ un punto en $\mathbb{R}^3$. Llamemos $F$ al campo vectorial dado por $F(\vec{x})=-\frac{x}{\|x\|^3}$.
Calcule el flujo del campo vectorial $F$ a trav\'es de la esfera de radio 2 centrada en el origen (orientada hacia afuera).

\end{frame}


\end{document}