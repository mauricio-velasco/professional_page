\documentclass[usepdftitle=false]{beamer}
\usetheme{Berkeley}



\title[]{Vectorial Virtual -- Taller 3, parte 2: Teoremas de Green y Stokes}

\date{Mayo 2020}

\begin{document}
\begin{frame}{Problema 6:}

Sea $u:\mathbb{R}^3\rightarrow \mathbb{R}$ una funci\'on escalar diferenciable. Sea $H=\nabla u$ el campo vectorial gradiente de $u$. 

\begin{enumerate}
\item Verifique que $\nabla \times H= \vec{0}$.
\item (V \'o F) Es verdad qu\'e $\nabla\cdot H = 0$ para cualquier funci\'on diferenciable $u$?
\item (V \'o F) Si $F$ es un campo vectorial con $\nabla \times F=0$ entonces $F$ es conservativo.

\end{enumerate}

\end{frame}

\begin{frame}{Problema 7: Teorema de Green}

\begin{enumerate}
\item Considere el campo vectorial 
\[F(x,y)=(22y+2x\sin(y)+17\sin(x),x^2\cos(y)+13y^{200\sin(y)})\]

Calcule el trabajo realizado por $F$ a lo largo del tri\'angulo con v\'ertices $(0,0)$, $(2\pi,0)$ y $(2\pi,2\pi)$ en ese orden.

\item Considere el campo vectorial \[G(x,y)=(3xy^2, 3x^2).\] 

Calcule el trabajo realizado por $G$ a lo largo de la frontera del rect\'angulo $0\leq x\leq 2$, $0\leq y\leq 3$ en direcci\'on de las manecillas del reloj.

\end{enumerate}

\end{frame}


\begin{frame}{Problema 8: Teorema de Green y el c\'alculo de \'areas} 

\begin{enumerate}
\item Sea $F(x,y)=(0,x)$. Utilice el Teorema de Green para demostrar que, para cualquier curva simple cerrada $\sigma$ positivamente orientada 
\[\int_{\sigma} Fd\vec{s} = Area(D)\]
Donde $D$ es la region encerrada por $\sigma$.
\item Sea $r(t)$ la curva parametrizada para $-1\leq t\leq 1$ por:
\[r(t)=\left(\frac{\sin(\pi t)^2}{t}, t^2-1\right)\]
\begin{enumerate}
\item Dibuje la curva (puede ayudarse con software)
\item Calcule el \'area encerrada por la curva usando la parte $(1)$.
\end{enumerate}

Esta es la idea detr\'as de un {\it plan\'imetro}
\url{https://en.wikipedia.org/wiki/Planimeter}

\end{enumerate}
 
 
 
 
\end{frame}

\begin{frame}{Problema 9: Teorema de Stokes}
El plano $z=x+4$ y el cilindro $x^2+y^2=4$ se intersectan en una curva $C$ orientada en contra de las manecillas del reloj cuando la vemos desde arriba. Calcule $\int_CFds$ donde $F$ es el campo vectorial dado por
\[F(x,y,z) = \left(x^3+2y,\sin(y)+z,x+\sin(z^2)\right)\] 
\end{frame}

\begin{frame}{Problema 10: Teorema de Stokes}
Sea $C$ la curva orientada parametrizada por 
\[r(t)=\left(\cos(t), \sin(t), 8-\cos^2(t)-\sin (t)\right)\] para $0\leq t\leq 2\pi$ y sea $F$ el campo vectorial dado por
\[F(x,y,z)=\left(z^2-y^2, -2xy^2, e^{\sqrt{z}}\cos(z)\right).\]
Calcule el trabajo realizado por $F$ a lo largo de $\sigma$. 
 
 
\end{frame}



\end{document}