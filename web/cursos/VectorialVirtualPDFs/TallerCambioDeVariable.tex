\documentclass[usepdftitle=false]{beamer}
\usetheme{Berkeley}



\title[]{Taller Vectorial Virtual parte 3: Cambio de variable.}

\date{Marzo 2020}

\begin{document}

\begin{frame}
  \titlepage
\end{frame}

\begin{frame}{Ejercicio 10}

Un plato plano tiene forma de un cuarto de anillo. M\'as precisamente esta acotado por los c\'irculos $x^2+y^2=4$, $x^2+y^2=9$ y las rectas $x=0$ y $y=0$. La densidad del plato esta descrita por la funci\'on $\rho(x,y)=x+y$, medida en $Kg/m^2$. Calcule la masa total del plato.
 \end{frame}

\begin{frame}{Ejercicio 11}
\begin{enumerate}
\item Dibuje la curva cartesiana cuya ecuaci\'on en coordenadas polares esta dada por $r=\cos(2\theta)$. Esta curva se llama rosa de cuatro p\'etalos.

\item Use el Teorema del cambio de variable para c\'alcular el \'area de un p\'etalo de la rosa (i.e. para $-\frac{\pi}{4}\leq \theta\leq \frac{\pi}{4}$)
\end{enumerate}

\end{frame}

\begin{frame}{Ejercicio 12}

\begin{enumerate}
\item Reescriba la integral iterada $\int_0^2\int_{-\sqrt{4-y^2}}^{\sqrt{4-y^2}}xydxdy$ como una integral en coordenadas polares.
\item Calcule el valor de la integral.
\end{enumerate}
\end{frame}


\begin{frame}{Ejercicio 13}

Sea $P$ el pol\'igono con v\'ertices $(0,0)$,$(3,1)$,$(1,2)$ y $(4,3)$.

\begin{enumerate}
\item Verif\'ique que $P$ es un paralelogramo.
\item Encuentre una transformaci\'on lineal $T$ que env\'ie el cuadrado $[0,1]\times [0,1]$ en $P$.
\item Use la parte $(b)$ para calcular 
\[\iint_P x^2+y^2 dA\]
\end{enumerate}


\end{frame}


\begin{frame}{Ejercicio 14}

Considere la transformaci\'on $T$ dada por
\[\begin{cases}
x=3u\\
y=2v
\end{cases}\]
\begin{enumerate}

\item Cu\'al es la im\'agen bajo $T$ del disco de radio uno centrado en $u=0,v=0$? Ll\'ame $D$ a esta regi\'on.
\item Use el teorema del cambio de variable para calcular el \'area de $D$.

\end{enumerate}



\end{frame}
\end{document}

\begin{frame}{Ejercicio 12}
\begin{enumerate}
\item Encuentre las coordenadas cartesianas $(x,y,z)$ del punto con coordenadas cil\'indricas $r=2$, $\theta =  \frac{2\pi}{3}$ y $z=1$.


\item Encuentre las coordenadas cartesianas $(x,y,z)$ del punto con coordenadas esf\'ericas $\rho=2$, $\theta =\frac{\pi}{4}$ y $ \phi =\frac{\pi}{3}$.

\item Encuentre las coordenadas cil\'indricas $(r,\theta, z)$ y esf\'ericas $(\rho,\theta,\phi)$ del punto con coordenadas rectangulares $(3,-3,7)$.

 
\end{enumerate}
\end{frame}



\end{document}

