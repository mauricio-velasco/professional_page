\documentclass[12pt]{article}
\oddsidemargin 0.0in
\topmargin 0.0in
\headheight 0.0in
\textwidth 7.0 in
\usepackage{amsmath}
\usepackage{amsthm}
\usepackage {amssymb}
\usepackage {stmaryrd}
\usepackage {graphicx}
\usepackage{graphics}
\usepackage[all]{xy}
\usepackage{amsrefs}

\newtheorem{theorem}{Theorem}
\newtheorem{cor}{Corollary}
\newtheorem{lemma}{Lemma}
\newtheorem{definition}{Definition}
\newtheorem{construction}{Construction}
\newtheorem{remark}{Remark}
\newtheorem{ex}{Example}
\newcommand{\RR}{\mathbb{R}}

\begin{document}
\begin{center}
\begin{Large}
{\bf C\'alculo Vectorial -- Parcial I}\\ 
\end{Large}
{Jueves 5 de Febrero de 2020.}
\bigskip
\\
\begin{tabular}{ll}
Nombre: & \dots\dots\dots\dots\dots\dots\dots\dots\dots\dots\dots\dots\dots\dots\dots\dots\dots\dots\\
\end{tabular}
\bigskip

\begin{large}

\bigskip
{\bf INSTRUCCIONES -- LEA ESTO ANTES DE EMPEZAR}
\end{large}
\end{center}
\begin{itemize}
\item{ Este examen tiene 4 problemas.}
\item{Muestre su trabajo. Para recibir todo el credito debe mostrar su razonamiento y los pasos que lo llevaron a la respuesta final y estos deben ser escritos claramente. Si necesita m\'as espacio escriba en la parte de atras del ejercicio anterior pero aseg\'urese de identificar claramente a que ejercicio corresponde cada pagina.}
\item{Este es un examen individual y con libro cerrado. Su Celular debe estar {\bf apagado} (si no puede apagarlo por motivos de urgencia mayor por favor comun\'iquelo a su profesor).}

\item{Este examen tiene una duracion de 80 mins.}
\end{itemize}
Se espera integridad academica de todos los estudiantes. Entendiendo esto, declaro que no voy a dar, usar o recibir ayuda no autorizada durante este examen.\\
\bigskip
\bigskip
\bigskip
\noindent Firma del estudiante:\\
\hline
\begin{center}
\begin{large}
\bigskip 
\begin{tabular}{|l|c|c|c|c|c|c|c|c|c|c|c}
\hline
Problema $\sharp$ & 1. & 2. & 3. & 4. & TOTAL\\
\hline
Puntos ganados & & & & &\\
\hline 
\end{tabular}
\end{large}
\end{center}
\newpage

\begin{enumerate}
\item {\bf [12 pts]} Sea $f(x,y)=e^{xy}+\sin(x+y)$.
\begin{enumerate}
\item  Verifique rigurosamente que $f(x,y)$ es diferenciable en todos los puntos $(x,y)\in \RR^2$.
\item Encuentre la funci\'on $\ell(x,y)$ que mejor aproxima a $f(x,y)$ cerca de $(0,0)$.
\item Encuentre la ecuaci\'on del plano tangente a la gr\'afica de $f(x,y)$ en $(0,0,1)$ y dibuje este plano.
\end{enumerate}

\newpage

\item {\bf [14 pts]} La temperatura de los puntos $(x,y,z)$ de $\RR^3$ esta dada por la funci\'on
\[T(x,y,z)= xyz.\] Un insecto camina sobre la esfera de ecuaci\'on $x^2+y^2+z^2=3$. Encuentre las temperaturas m\'inima y m\'axima que el insecto puede experimentar y los puntos de $\RR^3$ donde estas temperaturas se alcanzan.

\newpage
\item {\bf [12 pts]} \begin{enumerate}
\item Enuncie de manera precisa el "Teorema del gradiente" visto en clase.
\item Encuentre la ecuaci\'on del plano tangente a la colecci\'on de soluciones $(x,y,z)$ de la ecuaci\'on
\[xyz+x^2+y^2+z^2=4\]
en el punto $(1,1,1)$.
\item Justifique su soluci\'on de la parte $(b)$ utilizando el Teorema de la parte $(a)$.
\end{enumerate}


\newpage

\item {\bf [12 pts]} {\bf Verdadero o Falso:} En los siguientes ejercicios marque V si el enunciado es Verdadero y F si el enunciado es falso. No es necesario escribir la justificaci\'on de su respuesta. ESCRIBA SUS RESPUESTAS EN LA TABLA QUE APARECE A CONTINUACION.
\[
\begin{array}{|c|cc|}
\hline
a & & \\
\hline
b & & \\
\hline
c & & \\
\hline
d & & \\
\hline
\end{array}
\]



\begin{enumerate}

\item Si $\sigma(t)=(x(t),y(t),z(t))$ es una curva parametrizada que para todo $t$ esta contenida en la esfera de ecuaci\'on $x^2+y^2+z^2=100$ entonces los vectores $\sigma(t)$ y $\sigma'(t)$ son perpendiculares para todo $t$.


\item Sea $U:\RR^2\rightarrow \RR$ una funci\'on escalar diferenciable y defina $W(r,\theta)=U(r\cos\theta, r\sin\theta)$. Entonces para todo $r_0$ y $\theta_0$ tenemos que
\[\frac{\partial W}{\partial \theta}(r_0,\theta_0) = \frac{\partial U}{\partial x}(r_0\cos\theta_0, r_0\sin\theta_0)\left(-r_0\sin\theta_0\right)+ \frac{\partial U}{\partial y}(r_0\cos\theta_0, r_0\sin\theta_0)\left(r_0\cos\theta_0\right)\]

\item Sea $\sigma(t)$ una curva parametrizada en el plano. Si para todo $t$ tenemos $\|\sigma'(t)\|=10$ entonces $\sigma(t)$ parametriza una recta.

\item Existe un valor del n\'umero real $c$ para el cual la funci\'on $h(x,y)$ definida abajo es continua en $(0,0)$ 
\[ h(x,y)=\begin{cases}
\frac{xy}{x^2+y^2}\text{, si $(x,y)\neq (0,0)$}\\
c\text{, $(x,y) = (0,0)$}
\end{cases}\]

\end{enumerate}
\end{enumerate}
 
\end{document}
