\documentclass[notheorems]{beamer}
%\usetheme{Berkeley}
\newtheorem{theorem}{Theorem.}
\newtheorem{lemma}[theorem]{Lemma.}
\newtheorem{corollary}[theorem]{Corollary.}
\newtheorem{definition}[theorem]{Definition.}
\newtheorem{example}[theorem]{Example:}
\newtheorem{problem}[theorem]{Problem.}

\newtheorem{remark}[theorem]{Remark.}
\newtheorem{question}[theorem]{Question.}
\newtheorem{resalt}[theorem]{}
\newtheorem{conjecture}{Conjecture}

\usetheme{Copenhagen}

\newcommand{\pp}{\mathbb{P}}
\newcommand{\PP}{\mathbb{P}}

\newcommand{\zz}{\mathbb{Z}}
\newcommand{\rr}{\mathbb{R}}
\newcommand{\RR}{\mathbb{R}}
\newcommand{\R}{\mathbb{R}}

\newcommand{\CC}{\mathbb{C}}

\newcommand{\qq}{\mathbb{Q}}
\newcommand{\FF}{\mathbb{F}}
\newcommand{\ZZ}{\mathbb{Z}}

\newcommand{\qp}{\rm qp}
\newcommand{\codim}{\rm codim}

\usepackage{tikz}

\definecolor{cof}{RGB}{219,144,71}
\definecolor{pur}{RGB}{186,146,162}
\definecolor{greeo}{RGB}{91,173,69}
\definecolor{greet}{RGB}{52,111,72}


\usepackage{graphicx}
\usepackage{tikz}
\usepackage{tikz-cd}
\usetikzlibrary{cd}

\begin{document}
%Ejercicios de cambios de variable

\begin{frame}{Teorema Fund Integrales Linea (VA)}

Sea $u(x,y,z)=x^2+y^3+z^3$. Calcule el trabajo realizado por el campo vectorial $H(x,y,z)=\nabla u(x,y,z)$ a lo largo de la curva parametrizada $\sigma(t)=\left(\sin(t), \cos(t), cos(t/2)\right)$ con $0\leq t\leq 2\pi$.

\begin{enumerate}
\item $-2$
\item $2$
\item $-3$
\item $3$
\item $0$
\end{enumerate}


\end{frame}

\begin{frame}{Teorema Fund Integrales Linea (VB)}

Sea $u(x,y,z)=x^2+y^3+z^3$. Calcule el trabajo realizado por el campo vectorial $H(x,y,z)=\nabla u(x,y,z)$ a lo largo de la curva parametrizada $\sigma(t)=\left(\sin(t), \cos(t), 2cos(t)\right)$ con $0\leq t\leq 2\pi$.

\begin{enumerate}
\item $0$
\item $-2$
\item $2$
\item $-3$
\item $3$
\end{enumerate}


\end{frame}

\begin{frame}{Teorema Fund Integrales Linea (VC)}

Sea $u(x,y,z)=x^2+2y^3+z^2$. Calcule el trabajo realizado por el campo vectorial $H(x,y,z)=\nabla u(x,y,z)$ a lo largo de la curva parametrizada $\sigma(t)=\left(\sin(t), \cos(t/2), 2cos(t)\right)$ con $0\leq t\leq 2\pi$.

\begin{enumerate}
\item $-4$
\item $4$
\item $-3$
\item $3$
\item $0$
\end{enumerate}

\end{frame}

\begin{frame}{Teorema Fund Integrales Linea (VD)}
Sea $u(x,y,z)=3x^3+y^2+z^2$. Calcule el trabajo realizado por el campo vectorial $H(x,y,z)=\nabla u(x,y,z)$ a lo largo de la curva parametrizada $\sigma(t)=\left(\cos(t/2), \cos(t), 2cos(t)\right)$ con $0\leq t\leq 2\pi$.

\begin{enumerate}
\item $-6$
\item $6$
\item $-8$
\item $8$
\item $0$
\end{enumerate}



\end{frame}


\begin{frame}{Integrales de superficie (VA)}
Sea $H(x,y,z)=(0,0,z)$. Calcule el flujo de $H$ a trav\'es de la superficie cuadrada contenida en el plano $z=2$ cuyos puntos satisfacen $-2\leq x\leq 2$, $-2\leq y\leq 2$ orientada hacia arriba. 

\begin{enumerate}
\item $8$
\item $4$ 
\item $2$
\item $0$
\item $-2$
\end{enumerate}

\end{frame}

\begin{frame}{Integrales de superficie (VB)}
Sea $H(x,y,z)=(5x,0,0)$. Calcule el flujo de $H$ a trav\'es de la superficie cuadrada contenida en el plano $x=2$ cuyos puntos satisfacen $-2\leq y\leq 2$, $-2\leq z\leq 2$ orientada en la direcci\'on positiva del eje $x$. 

\begin{enumerate}
\item $40$
\item $60$ 
\item $20$
\item $10$
\item $0$
\end{enumerate}

\end{frame}
\begin{frame}{Integrales de superficie (VC)}
Sea $H(x,y,z)=(0,0,2z)$. Calcule el flujo de $H$ a trav\'es de la superficie cuadrada contenida en el plano $z=3$ cuyos puntos satisfacen $-2\leq x\leq 2$, $-2\leq y\leq 2$ orientada hacia arriba. 

\begin{enumerate}
\item $32$
\item $24$ 
\item $8$
\item $-16$
\item $-4$
\end{enumerate}

\end{frame}

\begin{frame}{Integrales de superficie (VD)}
Sea $H(x,y,z)=(3x,0,0)$. Calcule el flujo de $H$ a trav\'es de la superficie cuadrada contenida en el plano $x=1$ cuyos puntos satisfacen $-2\leq y\leq 2$, $-2\leq z\leq 2$ orientada en la direcci\'on positiva del eje $x$. 

\begin{enumerate}
\item $12$
\item $16$ 
\item $0$
\item $-12$
\item $-16$
\end{enumerate}

\end{frame}


\end{document}

