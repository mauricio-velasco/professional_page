\documentclass[usepdftitle=false]{beamer}
\usetheme{Berkeley}
\begin{document}

\begin{frame}{Problema 1. Tema A}

Cambie el orden de integraci\'on y eval\'ue la siguiente integral 
$\int_0^{\sqrt{\pi}}\int_y^{\sqrt{\pi}}\cos(x^2)dxdy=$ El resultado es:

\begin{enumerate}
\item
\item
\item
\item
\item

\end{enumerate}


\end{frame}

\begin{frame}{Problema 1. Tema B}

Cambie el orden de integraci\'on y eval\'ue la siguiente integral 
$\int_0^{1}\int_{\sqrt{x}}^{1}e^{y^3}dydx=$. El resultado es:


\begin{enumerate}
\item
\item
\item
\item
\item
\end{enumerate}

\end{frame}

\begin{frame}{Problema 2. Tema A}

La funci\'on de densidad de petr\'oleo en una regi\'on esta dada por $\rho(x,y,z)=\begin{cases}
e^{(x+y+z)}\text{, si $x,y,z\geq 0$}\\
0\text{, de lo contrario}
\end{cases}$
( en Toneladas / $m^3$). Cu\'al de los siguientes n\'umeros es la masa total del petr\'oleo contenido en la regi\'on encerrada por los planos $x+y+z=1$, $x=0$, $y=0$ y $z=0$?
\begin{enumerate}
\item
\item
\item
\item
\item
\end{enumerate}
\end{frame}

\begin{frame}{Problema 2. Tema B}

La funci\'on de densidad de petr\'oleo en una regi\'on esta dada por $\rho(x,y,z)=\begin{cases}
e^{(x-y+z)}\text{, si $x,y,z\geq 0$}\\
0\text{, de lo contrario}
\end{cases}$
( en Toneladas / $m^3$). Cu\'al de los siguientes n\'umeros es la masa total del petr\'oleo contenido en la regi\'on encerrada por los planos $x+y+z=1$, $x=0$, $y=0$ y $z=0$?
\begin{enumerate}
\item
\item
\item
\item
\item
\end{enumerate}
\end{frame}

\begin{frame}{Problema 3. Tema A}
Sea $E$ el s\'olido que esta debajo del paraboloide $z=10-x^2-y^2$ y encima del plano $z=0$. Cu\'al es el valor de $\iiint_E (x^3+x^2y)dV=$
\end{frame}

\begin{frame}{Problema 3. Tema B}
Sea $E$ el s\'olido que esta debajo del paraboloide $z=5-x^2-y^2$ y encima del plano $z=0$. Cu\'al es el valor de $\iiint_E (y^3+x^2y)dV=$
\end{frame}


\begin{frame}{Problema 4. Tema A}
Calcule el volumen de la regi\'on $\frac{x^2}{2^2}+\frac{y^2}{3^2}+\frac{z^2}{4^2}\leq 1$. (Ayuda: Use le cambio de variable $(x,y,z)=(2u,3v,4w)$). Ese volumen es:

\begin{enumerate}
\item
\item
\item
\item
\item
\end{enumerate}

\end{frame}

\begin{frame}{Problema 4. Tema B}
Calcule el volumen de la regi\'on $\frac{x^2}{3^2}+\frac{y^2}{4^2}+\frac{z^2}{5^2}\leq 1$. (Ayuda: Use le cambio de variable $(x,y,z)=(3u,4v,5w)$). Ese volumen es:

\begin{enumerate}
\item
\item
\item
\item
\item
\end{enumerate}

\end{frame}


\begin{frame}{Problema 5. Tema A}

Diga cu\'al de las siguientes funciones es un potencial escalar para el campo vectorial $H(x,y,z)=$

\begin{enumerate}
\item 
\item
\end{enumerate}

\end{frame}


\begin{frame}{Problema 5. Tema B}

Sea $H(x,y,z)=\nabla u$ donde $u(x,y,z)=\sin(xyz)$. Sea $\sigma$ la curva parametrizada dada por $\sigma(t)=(e^t,t,t^2)$ para $0\leq 1\leq 1$. Cu\'al de las siguientes cantidades es igual al valor de la integral de l\'inea
$\int_{\sigma} H\cdot ds=$


\end{frame}

\end{document}

\end{document}
